%%%%%%%%%%%%%%%%%%%%%%%%%%%%%%%%%%
%%%%% Language configuration %%%%%
%%%%%%%%%%%%%%%%%%%%%%%%%%%%%%%%%%
%%%%% The language of the resume can be set below.
%%%%%
%%%%% All languages supported by the "babel" package can be used.
%%%%% Note that this document uses babel's "classical"
%%%%% (also called "ldf") language names.
%%%%%
%%%%% A list of supported languages can be found in chapter 1.6 of
%%%%% the following PDF:
%%%%% https://ctan.net/macros/latex/required/babel/base/babel.pdf
%%%%%

% Language of the document
\newcommand{\resumeLanguage}{english}


%%%%%%%%%%%%%%%%%%%%%%%%%%%%%%%
%%%%% Contact information %%%%%
%%%%%%%%%%%%%%%%%%%%%%%%%%%%%%%
%%%%% Define your contact information below.
%%%%%
%%%%% Note that some attributes are optional. They can be omitted
%%%%% by setting them to an empty string, e.g.:
%%%%%
%%%%%     \newcommand{\unusedSetting}{}
%%%%%

% Your name
\newcommand{\resumeName}{John Doe}

% Your postal address
% (Optional, set to an empty string to remove)
\newcommand{\resumeAddress}{Deer Ave 12, Cappuccino Springs, XY 12345}

% Your email address
\newcommand{\resumeEmail}{mail@example.com}
% Your phone number
\newcommand{\resumePhone}{+1234567890}
% Your phone number with spaces, to improve legibility
\newcommand{\resumePhonePrint}{+1 234 567890}

% Full link to your website
% (Optional, set to an empty string to remove)
\newcommand{\resumeWebsite}{https://example.com/website}
% Short website link that will be visible in the PDF
\newcommand{\resumeWebsitePrint}{example.com/website}


%%%%%%%%%%%%%%%%%%%
%%%%% Picture %%%%%
%%%%%%%%%%%%%%%%%%%
%%%%% In the top right corner of the resume, a picture can be added.
%%%%% All common image formats are supported.
%%%%%
%%%%% To turn off the picture, set \mainPicture to an empty string:
%%%%%
%%%%%     \newcommand{\mainPicture}{}
%%%%%

% File path of the main picture, relative to the main folder. 
% (Optional, set to an empty string to turn off the picture.)
\newcommand{\mainPicture}{fig/photo.png}
% The picture will be scaled to the following width:
\newcommand{\mainPictureWidth}{35mm}


%%%%%%%%%%%%%%%%%%%%%
%%%%% Signature %%%%%
%%%%%%%%%%%%%%%%%%%%%
%%%%% A signature line can be added to the bottom right corner.
%%%%% There are three options:
%%%%%
%%%%% - A signature line with a signature
%%%%% - A signature line without signature (to sign by hand)
%%%%% - Nothing
%%%%%
%%%%% When adding a printed signature, a file with a transparent
%%%%% background must be used. PDF and PNG files have been 
%%%%% tested successfully.
%%%%%

% Comment out the following line to completely remove the signature line
\newcommand{\includeSignature}{}

%%% Signature picture settings %%%
% File path of the signature picture, relative to the main folder.
% The signature can be a PNG or PDF file with a transparent background.
% (Optional, set to an empty string to turn off the picture.)
\newcommand{\signaturePicture}{fig/signature.pdf}
% Adjust the height of the printed signature
\newcommand{\signatureHeight}{2cm}
% Move the printed signature to the desired position
\newcommand{\signatureShiftUp}{-0.25cm}
\newcommand{\signatureShiftRight}{0.3cm}

%%% Signature line settings %%%
% The location of the signature
\newcommand{\signatureCity}{Electri City}
% The date of the signature
% (Leave empty to use the current date)
\newcommand{\signatureDate}{}


%%%%%%%%%%%%%%%%%%%%%%%%%%%%%%%
%%%%% Color configuration %%%%%
%%%%%%%%%%%%%%%%%%%%%%%%%%%%%%%
%%%%% Colors known to the xcolors package can be used:
%%%%%
%%%%%     \newcommand{\exampleSetting}{blue}
%%%%%
%%%%% This includes the dvips colors:
%%%%%
%%%%%     \newcommand{\exampleSetting}{Apricot}
%%%%%
%%%%% You can also define new color names in this file with \definecolor:
%%%%%
%%%%%     \definecolor{myOrange}{RGB}{255,127,0}
%%%%%     \newcommand{\exampleSetting}{myOrange}
%%%%%
%%%%% For a list of available colors and more infos, see:
%%%%% https://en.wikibooks.org/wiki/LaTeX/Colors#Predefined_colors
%%%%%

% Default color for text that doesn't have another color
\newcommand{\defaultTextColor}{black}

%%% Colors used in the header %%%
% Name at the very top
\newcommand{\nameColor}{\defaultTextColor}
% Hyperlinks in the header (only if \useColoredLinks is "true", see below)
\definecolor{DarkBlue}{RGB}{0,0,148}
\newcommand{\urlColor}{DarkBlue}
% Frames displayed around hyperlinks (only if \useColoredLinks is "false", see below)
\newcommand{\urlFrameColor}{white}
% Symbols (envelope etc.) next to the hyperlinks
\newcommand{\headerSymbolColor}{\defaultTextColor}

%%% Main text colors %%%
% Big titles
\newcommand{\bigTitleColor}{gray}
% Smaller titles
\newcommand{\smallTitleColor}{\defaultTextColor}
% Text in parantheses to the right of small titles
\newcommand{\topItalicsColor}{darkgray}
% Italic text directly below small titles
\newcommand{\bottomItalicsColor}{darkgray}

%%% Line colors %%%
% Line around the header photo
\newcommand{\photoBorderColor}{black}
% Lines below the main titles
\newcommand{\titleLineColor}{black}
% Line below the signature
\newcommand{\signatureLineColor}{black}


%%%%%%%%%%%%%%%%%%%%%%%%%%%%%
%%%%% URL configuration %%%%%
%%%%%%%%%%%%%%%%%%%%%%%%%%%%%
%%%%% URLs can be visually distinguished from normal text in one of two ways:
%%%%%
%%%%% - URLs use colored text. This ensures a consistent presentation when
%%%%%   the file is digitally viewed, but can lead to reduced print quality if the
%%%%%   file is printed with a black-and-white printer.
%%%%% - URLs use black text, with a colored frame that is only visible when the
%%%%%   file is viewed digitally. This ensures optimal print quality, while
%%%%%   highlighting the URLs only when they can actually be clicked. However, the
%%%%%   colored frames may look unappealing compared to colored text.
%%%%%
%%%%% For more information, see:
%%%%% https://en.wikibooks.org/wiki/LaTeX/Hyperlinks#Customization
%%%%%

%% Style used to display hyperlinks
% Set to "true" to make hyperlinks use colored text.
% Set to "false" to display frames around hyperlinks instead.
\newcommand{\useColoredLinks}{false}
% Width of the frames around hyperlinks
\newcommand{\urlFrameWidth}{0.1}
% Margin of the frames around hyperlinks
\newcommand{\urlFrameMargin}{0.1em}
